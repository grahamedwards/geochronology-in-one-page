\documentclass[12pt, letterpaper,twocolumn]{article}
\usepackage[margin=.75in]{geometry}
\usepackage{graphicx} % Required for inserting images
\usepackage{amsmath}
\pagestyle{empty}

\usepackage[tiny,noindentafter]{titlesec}
\titlespacing{\section}{0ex}{2ex}{0ex} % left, before-sep, after-sep, {right-sep}
\titleformat{\section}[hang]{\bfseries}{Part \thesection:}{1em}{}

\setlength{\parskip}{1.5ex}\setlength{\parindent}{0ex}
\setlength{\columnsep}{3em}


\usepackage{titling}\setlength{\droptitle}{-60pt} % flush the title with the upper margin.

\addtolength{\jot}{1ex} % Add a little extra room between equations in align.

\title{
    Radiometric dating equations % Title
    (in one page) \vspace*{-1ex}}\date{\vspace*{-6ex}}

\author{
    Graham H.~Edwards % Author(s)
    }
     

\begin{document}
\maketitle \thispagestyle{empty}

\section{The age equation}

Begin with the observation that the rate of \emph{decay} of a radioactive isotope with respect to time ($dN/dt$) is directly proportional to its abundance $N$ at any given moment:
\begin{equation}\label{eq:dNdt}
    \frac{dN}{dt} = -\lambda N
\end{equation}

\dots assuming a constant of proportionality $\lambda$, which we'll call $N$'s ``decay constant.''

Now, separate the derivatives in Eq.~\ref{eq:dNdt} and calculate their definite integrals over the time interval $\left[0,t\right]$.
\begin{subequations}\label{eq:age}
    \begin{align*}
        \frac{1}{N}~dN &= -\lambda~dt
        \\
        \int_0^t \frac{1}{N}~dN & = \int_0^t -\lambda~dt
        \\
        \ln N - \ln N_o &= -\lambda t - -\lambda \cdot 0
        \\
        \ln \frac{N}{N_o} &= -\lambda t
        \\
        \frac{N}{N_o} &= e^{-\lambda t}
        \\
        N &= N_o~e^{-\lambda t} \tag{\ref{eq:age}}
    \end{align*}
\end{subequations}

\dots where $N$ is the abundance of the radioactive isotope at some moment in time $t$, and $N_o$ is its \emph{initial} abundance at time $t=0$.

Eq.~\ref{eq:age} is the radiometric ``age equation.''

\section{Using the age equation}
Start with Eq.~\ref{eq:age}, which describes the amount of radioactive isotope $N$ (with initial abundance $N_o$) remaining after time $t$, given the isotope's decay constant $\lambda$ (units of $t^{-1}$).

From this, calculate the abundance of radiogenic isotope $n*$ produced by the decay of $N$\dots
\begin{subequations}\label{eq:nNo}
\begin{align*}
    n* &= N_o - N 
\\
    n* &= N_o - N_o~e^{-\lambda t}
\\
    n* &= N_o \left( 1 - e^{-\lambda t} \right) \tag{\ref{eq:nNo}}
\end{align*}
\end{subequations}

This is lovely but requires knowledge of $N_o$, which is lost to time. 
All we can measure today is the value $N$.
So, let's rearrange Eq.~\ref{eq:age} to solve for $N_o$\dots
\begin{equation} \label{eq:No}
    N_o = \frac{N}{e^{-\lambda t}} = N~e^{\lambda t}
\end{equation}

\dots and substitute Eq.~\ref{eq:No} into Eq.~\ref{eq:nNo}\dots

\begin{subequations}\label{eq:nN}
\begin{align*}
     n* &= \left(N~e^{\lambda t}\right) \left( 1 - e^{-\lambda t}\right) 
     \\
     n* &= N \left( e^{\lambda t} - e^{\lambda t} e^{-\lambda t}\right) 
     \\
     n* &= N \left( e^{\lambda t} - 1 \right) \tag{\ref{eq:nN}}
\end{align*}
\end{subequations}

A natural system may have originally incorporated some initial amount of radiogenic isotope $n_o$, so  we calculate $n = n_o + {n*}$ as

\begin{equation}\label{eq:nfull}
    n = n_o + N \left( e^{\lambda t} - 1 \right)
\end{equation}

And \emph{finally}, because it is, in fact, quite hard to measure things as absolute abundances but easier to measure isotopic ratios, we often rewrite Eq.~\ref{eq:nfull} normalized to a stable isotope of the same element as $n$, which we'll call $^S n$.

\begin{equation}\label{eq:ratios}
    \frac{n}{^S n} = \frac{n_o}{^S n} + \frac{N}{^S n} \left( e^{\lambda t} - 1 \right)
\end{equation}

\end{document}
